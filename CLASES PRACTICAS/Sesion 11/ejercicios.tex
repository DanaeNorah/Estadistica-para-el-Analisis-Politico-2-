% Options for packages loaded elsewhere
\PassOptionsToPackage{unicode}{hyperref}
\PassOptionsToPackage{hyphens}{url}
%
\documentclass[
]{article}
\usepackage{lmodern}
\usepackage{amssymb,amsmath}
\usepackage{ifxetex,ifluatex}
\ifnum 0\ifxetex 1\fi\ifluatex 1\fi=0 % if pdftex
  \usepackage[T1]{fontenc}
  \usepackage[utf8]{inputenc}
  \usepackage{textcomp} % provide euro and other symbols
\else % if luatex or xetex
  \usepackage{unicode-math}
  \defaultfontfeatures{Scale=MatchLowercase}
  \defaultfontfeatures[\rmfamily]{Ligatures=TeX,Scale=1}
\fi
% Use upquote if available, for straight quotes in verbatim environments
\IfFileExists{upquote.sty}{\usepackage{upquote}}{}
\IfFileExists{microtype.sty}{% use microtype if available
  \usepackage[]{microtype}
  \UseMicrotypeSet[protrusion]{basicmath} % disable protrusion for tt fonts
}{}
\makeatletter
\@ifundefined{KOMAClassName}{% if non-KOMA class
  \IfFileExists{parskip.sty}{%
    \usepackage{parskip}
  }{% else
    \setlength{\parindent}{0pt}
    \setlength{\parskip}{6pt plus 2pt minus 1pt}}
}{% if KOMA class
  \KOMAoptions{parskip=half}}
\makeatother
\usepackage{xcolor}
\IfFileExists{xurl.sty}{\usepackage{xurl}}{} % add URL line breaks if available
\IfFileExists{bookmark.sty}{\usepackage{bookmark}}{\usepackage{hyperref}}
\hypersetup{
  pdftitle={Proceso del Analisis Factorial Exploratorio (EFA)},
  hidelinks,
  pdfcreator={LaTeX via pandoc}}
\urlstyle{same} % disable monospaced font for URLs
\usepackage[margin=1in]{geometry}
\usepackage{color}
\usepackage{fancyvrb}
\newcommand{\VerbBar}{|}
\newcommand{\VERB}{\Verb[commandchars=\\\{\}]}
\DefineVerbatimEnvironment{Highlighting}{Verbatim}{commandchars=\\\{\}}
% Add ',fontsize=\small' for more characters per line
\usepackage{framed}
\definecolor{shadecolor}{RGB}{248,248,248}
\newenvironment{Shaded}{\begin{snugshade}}{\end{snugshade}}
\newcommand{\AlertTok}[1]{\textcolor[rgb]{0.94,0.16,0.16}{#1}}
\newcommand{\AnnotationTok}[1]{\textcolor[rgb]{0.56,0.35,0.01}{\textbf{\textit{#1}}}}
\newcommand{\AttributeTok}[1]{\textcolor[rgb]{0.77,0.63,0.00}{#1}}
\newcommand{\BaseNTok}[1]{\textcolor[rgb]{0.00,0.00,0.81}{#1}}
\newcommand{\BuiltInTok}[1]{#1}
\newcommand{\CharTok}[1]{\textcolor[rgb]{0.31,0.60,0.02}{#1}}
\newcommand{\CommentTok}[1]{\textcolor[rgb]{0.56,0.35,0.01}{\textit{#1}}}
\newcommand{\CommentVarTok}[1]{\textcolor[rgb]{0.56,0.35,0.01}{\textbf{\textit{#1}}}}
\newcommand{\ConstantTok}[1]{\textcolor[rgb]{0.00,0.00,0.00}{#1}}
\newcommand{\ControlFlowTok}[1]{\textcolor[rgb]{0.13,0.29,0.53}{\textbf{#1}}}
\newcommand{\DataTypeTok}[1]{\textcolor[rgb]{0.13,0.29,0.53}{#1}}
\newcommand{\DecValTok}[1]{\textcolor[rgb]{0.00,0.00,0.81}{#1}}
\newcommand{\DocumentationTok}[1]{\textcolor[rgb]{0.56,0.35,0.01}{\textbf{\textit{#1}}}}
\newcommand{\ErrorTok}[1]{\textcolor[rgb]{0.64,0.00,0.00}{\textbf{#1}}}
\newcommand{\ExtensionTok}[1]{#1}
\newcommand{\FloatTok}[1]{\textcolor[rgb]{0.00,0.00,0.81}{#1}}
\newcommand{\FunctionTok}[1]{\textcolor[rgb]{0.00,0.00,0.00}{#1}}
\newcommand{\ImportTok}[1]{#1}
\newcommand{\InformationTok}[1]{\textcolor[rgb]{0.56,0.35,0.01}{\textbf{\textit{#1}}}}
\newcommand{\KeywordTok}[1]{\textcolor[rgb]{0.13,0.29,0.53}{\textbf{#1}}}
\newcommand{\NormalTok}[1]{#1}
\newcommand{\OperatorTok}[1]{\textcolor[rgb]{0.81,0.36,0.00}{\textbf{#1}}}
\newcommand{\OtherTok}[1]{\textcolor[rgb]{0.56,0.35,0.01}{#1}}
\newcommand{\PreprocessorTok}[1]{\textcolor[rgb]{0.56,0.35,0.01}{\textit{#1}}}
\newcommand{\RegionMarkerTok}[1]{#1}
\newcommand{\SpecialCharTok}[1]{\textcolor[rgb]{0.00,0.00,0.00}{#1}}
\newcommand{\SpecialStringTok}[1]{\textcolor[rgb]{0.31,0.60,0.02}{#1}}
\newcommand{\StringTok}[1]{\textcolor[rgb]{0.31,0.60,0.02}{#1}}
\newcommand{\VariableTok}[1]{\textcolor[rgb]{0.00,0.00,0.00}{#1}}
\newcommand{\VerbatimStringTok}[1]{\textcolor[rgb]{0.31,0.60,0.02}{#1}}
\newcommand{\WarningTok}[1]{\textcolor[rgb]{0.56,0.35,0.01}{\textbf{\textit{#1}}}}
\usepackage{graphicx,grffile}
\makeatletter
\def\maxwidth{\ifdim\Gin@nat@width>\linewidth\linewidth\else\Gin@nat@width\fi}
\def\maxheight{\ifdim\Gin@nat@height>\textheight\textheight\else\Gin@nat@height\fi}
\makeatother
% Scale images if necessary, so that they will not overflow the page
% margins by default, and it is still possible to overwrite the defaults
% using explicit options in \includegraphics[width, height, ...]{}
\setkeys{Gin}{width=\maxwidth,height=\maxheight,keepaspectratio}
% Set default figure placement to htbp
\makeatletter
\def\fps@figure{htbp}
\makeatother
\setlength{\emergencystretch}{3em} % prevent overfull lines
\providecommand{\tightlist}{%
  \setlength{\itemsep}{0pt}\setlength{\parskip}{0pt}}
\setcounter{secnumdepth}{-\maxdimen} % remove section numbering

\title{Proceso del Analisis Factorial Exploratorio (EFA)}
\author{}
\date{\vspace{-2.5em}}

\begin{document}
\maketitle

\hypertarget{ejercicio-1-base-de-datos-de-departamentos}{%
\section{Ejercicio 1: Base de datos de
departamentos}\label{ejercicio-1-base-de-datos-de-departamentos}}

\hypertarget{paso-0-seleccionamos-subdata}{%
\subsection{Paso 0: Seleccionamos
subdata}\label{paso-0-seleccionamos-subdata}}

Abrimos la base de datos:

\begin{Shaded}
\begin{Highlighting}[]
\KeywordTok{library}\NormalTok{(rio)}
\NormalTok{regiones<-}\StringTok{ }\KeywordTok{import}\NormalTok{(}\StringTok{"https://github.com/DataPolitica/salidas/raw/master/Data/regiones_total.xlsx"}\NormalTok{)}
\KeywordTok{head}\NormalTok{(regiones)}
\end{Highlighting}
\end{Shaded}

\begin{verbatim}
##   n    Región casos   casos_t fallecidos fallecidos_t poblacion poblacion_100k
## 1 1  Amazonas  9968 2627.4171        190     50.08118    379384        3.79384
## 2 2    Ancash 17314 1597.9415       1141    105.30503   1083519       10.83519
## 3 3  Apurímac  2448  603.3138         60     14.78710    405759        4.05759
## 4 4  Arequipa 30441 2201.5144       1106     79.98669   1382730       13.82730
## 5 5  Ayacucho  7407 1202.0916        253     41.05970    616176        6.16176
## 6 6 Cajamarca 13860 1033.5478        377     28.11310   1341012       13.41012
##   altura_media_capital pobreza vias_pavimentadas problema_salud_cronico
## 1                 2483    47.3           42.0695               42.41115
## 2                 3050    23.5           30.7078               41.37640
## 3                 2500    42.8           34.6249               40.20969
## 4                 2335     9.1           28.3488               33.07475
## 5                 2746    51.9           34.5617               29.67002
## 6                 2750    52.9           38.8711               28.28153
##   esperanza_vida  agua desague electrificacion medicos_colegiados PBI_percapita
## 1          71.65 51.84   36.69           73.67                262      6820.929
## 2          74.86 71.56   56.38           85.20               2741     16846.823
## 3          71.44 56.33   36.12           80.43                508     14738.951
## 4          77.13 72.47   65.85           89.98               2195     23096.513
## 5          72.01 66.99   45.35           80.94                560      7783.191
## 6          74.03 52.89   32.48           80.68                464      7081.692
##   presupuestopublico_percapita ejecución_inversion_publica PEA_informal
## 1                     446.2813                     72.9082      85.3501
## 2                     509.8968                     53.1842      80.3267
## 3                     519.3545                     61.7492      88.0408
## 4                     403.1003                     61.5159      65.2675
## 5                     462.4479                     67.7971      87.0986
## 6                     326.7535                     61.0216      89.3412
##   analfabetismo poblacion_secundaria
## 1        8.4242              41.2392
## 2        9.5630              51.4177
## 3       14.0318              43.6943
## 4        3.5449              68.1574
## 5       11.8598              46.1963
## 6       11.5018              35.7519
\end{verbatim}

Realizamos un subset y nos quedamos con las variables que utilizaremos.
Lo llamaremos \textbf{subdata}:

\begin{itemize}
\tightlist
\item
  nos aseguramos que usamos la variables que seleccionamos (agua,
  desgue, electrificacion,PBI\ldots)
\end{itemize}

\begin{Shaded}
\begin{Highlighting}[]
\NormalTok{subdata<-}\StringTok{ }\NormalTok{regiones[,}\KeywordTok{c}\NormalTok{(}\DecValTok{11}\NormalTok{, }\DecValTok{14}\OperatorTok{:}\DecValTok{16}\NormalTok{, }\DecValTok{18}\OperatorTok{:}\DecValTok{19}\NormalTok{)]}
\end{Highlighting}
\end{Shaded}

\begin{center}\rule{0.5\linewidth}{0.5pt}\end{center}

\hypertarget{paso-1-matriz-de-correlaciones}{%
\subsection{Paso 1: Matriz de
correlaciones}\label{paso-1-matriz-de-correlaciones}}

Generamos la matriz de correlaciones para identificar qué variables de
nuestra subdata están correlacionadas.

\begin{Shaded}
\begin{Highlighting}[]
\KeywordTok{library}\NormalTok{(polycor)}
\end{Highlighting}
\end{Shaded}

\begin{verbatim}
## Warning: package 'polycor' was built under R version 4.0.3
\end{verbatim}

\begin{Shaded}
\begin{Highlighting}[]
\NormalTok{matriz<-}\KeywordTok{hetcor}\NormalTok{(subdata)}
\NormalTok{matriz_corr<-}\StringTok{ }\NormalTok{matriz}\OperatorTok{$}\NormalTok{correlations}
\KeywordTok{library}\NormalTok{(ggcorrplot)}
\end{Highlighting}
\end{Shaded}

\begin{verbatim}
## Warning: package 'ggcorrplot' was built under R version 4.0.3
\end{verbatim}

\begin{verbatim}
## Loading required package: ggplot2
\end{verbatim}

\begin{Shaded}
\begin{Highlighting}[]
\KeywordTok{ggcorrplot}\NormalTok{(matriz_corr)}
\end{Highlighting}
\end{Shaded}

\includegraphics{ejercicios_files/figure-latex/unnamed-chunk-3-1.pdf}

Vemos las correlaciones significativas: Si puedes ver bloques
correlacionados hay esperanza de un buen analisis factorial.

-ver en el gaafico con cuadrados MAS INTENSOS

\begin{Shaded}
\begin{Highlighting}[]
\KeywordTok{ggcorrplot}\NormalTok{(matriz_corr,}
           \DataTypeTok{p.mat =} \KeywordTok{cor_pmat}\NormalTok{(matriz_corr),}
           \DataTypeTok{insig =} \StringTok{"blank"}\NormalTok{)}
\end{Highlighting}
\end{Shaded}

\includegraphics{ejercicios_files/figure-latex/unnamed-chunk-4-1.pdf}

\begin{Shaded}
\begin{Highlighting}[]
\CommentTok{#te sirve para redondear a dos decimales, pero si se corresolo "matriz_corr" es para sacar}
\KeywordTok{round}\NormalTok{(matriz_corr, }\DecValTok{2}\NormalTok{)}
\end{Highlighting}
\end{Shaded}

\begin{verbatim}
##                              vias_pavimentadas  agua desague electrificacion
## vias_pavimentadas                         1.00 -0.01    0.01           -0.21
## agua                                     -0.01  1.00    0.88            0.91
## desague                                   0.01  0.88    1.00            0.84
## electrificacion                          -0.21  0.91    0.84            1.00
## PBI_percapita                             0.32  0.44    0.60            0.39
## presupuestopublico_percapita              0.24  0.10    0.03            0.02
##                              PBI_percapita presupuestopublico_percapita
## vias_pavimentadas                     0.32                         0.24
## agua                                  0.44                         0.10
## desague                               0.60                         0.03
## electrificacion                       0.39                         0.02
## PBI_percapita                         1.00                         0.58
## presupuestopublico_percapita          0.58                         1.00
\end{verbatim}

\hypertarget{paso-2-diagnuxf3stico-de-nuestra-matruxedz-de-correlaciones}{%
\subsection{Paso 2: Diagnóstico de nuestra matríz de
correlaciones}\label{paso-2-diagnuxf3stico-de-nuestra-matruxedz-de-correlaciones}}

Primero, verificar si datos permiten factorizar:

-Test KMO: cuanto mas cerca a 1 implica relacion es alta. Tiene que ser
mayor que 0.5 - FIJARTE EN EL OVERALL MSA

\begin{Shaded}
\begin{Highlighting}[]
\KeywordTok{library}\NormalTok{(psych)}
\end{Highlighting}
\end{Shaded}

\begin{verbatim}
## 
## Attaching package: 'psych'
\end{verbatim}

\begin{verbatim}
## The following objects are masked from 'package:ggplot2':
## 
##     %+%, alpha
\end{verbatim}

\begin{verbatim}
## The following object is masked from 'package:polycor':
## 
##     polyserial
\end{verbatim}

\begin{Shaded}
\begin{Highlighting}[]
\KeywordTok{KMO}\NormalTok{(matriz_corr) }
\end{Highlighting}
\end{Shaded}

\begin{verbatim}
## Kaiser-Meyer-Olkin factor adequacy
## Call: KMO(r = matriz_corr)
## Overall MSA =  0.53
## MSA for each item = 
##            vias_pavimentadas                         agua 
##                         0.27                         0.58 
##                      desague              electrificacion 
##                         0.60                         0.70 
##                PBI_percapita presupuestopublico_percapita 
##                         0.45                         0.27
\end{verbatim}

Segundo, verificar si la matriz de correlaciones es adecuada. Para ello,
se tienen dos funciones:

\begin{itemize}
\tightlist
\item
  Test de Bartlett: H0: La matriz de correlacion es una matriz
  identidad. Buscamos rechazar la H0, por eso esperamos que sea
  signifitivo (False). Si no sale FALSE es grave. Matriz identidad
  (diagonal es 1, y lo demas es 0), solo una vairable se correlaciona
  consigo misma.
\end{itemize}

\begin{Shaded}
\begin{Highlighting}[]
\KeywordTok{cortest.bartlett}\NormalTok{(matriz_corr,}\DataTypeTok{n=}\KeywordTok{nrow}\NormalTok{(subdata))}\OperatorTok{$}\NormalTok{p.value}\OperatorTok{>}\FloatTok{0.05}
\end{Highlighting}
\end{Shaded}

\begin{verbatim}
## [1] FALSE
\end{verbatim}

\begin{itemize}
\tightlist
\item
  Test For Singular Square Matrix: H0: La matriz de correlacion es una
  matriz singular. Una matriz que no sea inversa Buscamos que sea False.
  SI sale TRUE no seguir
\end{itemize}

\begin{Shaded}
\begin{Highlighting}[]
\KeywordTok{library}\NormalTok{(matrixcalc)}
\end{Highlighting}
\end{Shaded}

\begin{verbatim}
## Warning: package 'matrixcalc' was built under R version 4.0.3
\end{verbatim}

\begin{Shaded}
\begin{Highlighting}[]
\KeywordTok{is.singular.matrix}\NormalTok{(matriz_corr)}
\end{Highlighting}
\end{Shaded}

\begin{verbatim}
## [1] FALSE
\end{verbatim}

\hypertarget{paso-3-identificamos-el-nuxfamero-recomendado-de-factores-y-solicitamos-el-efa}{%
\subsection{Paso 3: Identificamos el número recomendado de factores y
solicitamos el
EFA}\label{paso-3-identificamos-el-nuxfamero-recomendado-de-factores-y-solicitamos-el-efa}}

Determinar en cuantos factores o variables latentes podríamos
redimensionar la data. Vemos el número sugerido y también el gráfico.

\begin{Shaded}
\begin{Highlighting}[]
\CommentTok{#insumo basico para hacer el analisis factorial (DIARGRAMA DE SEDIMENTACION nos dice cuantos factores deberiamos calcular)}
\KeywordTok{fa.parallel}\NormalTok{(subdata, }\DataTypeTok{fm =} \StringTok{'ML'}\NormalTok{, }\DataTypeTok{fa =} \StringTok{'fa'}\NormalTok{)}
\end{Highlighting}
\end{Shaded}

\includegraphics{ejercicios_files/figure-latex/unnamed-chunk-9-1.pdf}

\begin{verbatim}
## Parallel analysis suggests that the number of factors =  2  and the number of components =  NA
\end{verbatim}

Solicitamos el número de factores. Considerar si se presentan mensajes
de alerta.

\begin{Shaded}
\begin{Highlighting}[]
\KeywordTok{library}\NormalTok{(GPArotation)}
\end{Highlighting}
\end{Shaded}

\begin{verbatim}
## Warning: package 'GPArotation' was built under R version 4.0.3
\end{verbatim}

\begin{Shaded}
\begin{Highlighting}[]
\NormalTok{factorial <-}\StringTok{ }\KeywordTok{fa}\NormalTok{(subdata,}\DataTypeTok{nfactors =} \DecValTok{2}\NormalTok{, }\DataTypeTok{cor =} \StringTok{'mixed'}\NormalTok{, }\DataTypeTok{rotate =} \StringTok{"varimax"}\NormalTok{,}\DataTypeTok{fm=}\StringTok{"minres"}\NormalTok{)}
\end{Highlighting}
\end{Shaded}

\begin{verbatim}
## Warning in fa.stats(r = r, f = f, phi = phi, n.obs = n.obs, np.obs = np.obs, :
## The estimated weights for the factor scores are probably incorrect. Try a
## different factor score estimation method.
\end{verbatim}

\begin{verbatim}
## Warning in fac(r = r, nfactors = nfactors, n.obs = n.obs, rotate = rotate, : An
## ultra-Heywood case was detected. Examine the results carefully
\end{verbatim}

\begin{Shaded}
\begin{Highlighting}[]
\CommentTok{#correlacion mixta, rotacion}
\end{Highlighting}
\end{Shaded}

\hypertarget{paso-4-visualizamos-el-efa-solicitado}{%
\subsection{Paso 4: Visualizamos el EFA
solicitado}\label{paso-4-visualizamos-el-efa-solicitado}}

Vemos el resultado inicial: - mayor sea el numero mas aporta al factor.
uando una vairbale tiene loadings en dos factores quiere decir que
aporta a los dos factores

\begin{Shaded}
\begin{Highlighting}[]
\KeywordTok{print}\NormalTok{(factorial}\OperatorTok{$}\NormalTok{loadings)}
\end{Highlighting}
\end{Shaded}

\begin{verbatim}
## 
## Loadings:
##                              MR1    MR2   
## vias_pavimentadas                    0.424
## agua                          0.934       
## desague                       0.916  0.149
## electrificacion               0.964       
## PBI_percapita                 0.468  0.888
## presupuestopublico_percapita         0.605
## 
##                  MR1   MR2
## SS loadings    2.869 1.369
## Proportion Var 0.478 0.228
## Cumulative Var 0.478 0.706
\end{verbatim}

Vemos el resultado mejorado: Cuando logramos que cada variable se vaya a
un factor, tenemos una estructura simple.

\begin{Shaded}
\begin{Highlighting}[]
\KeywordTok{print}\NormalTok{(factorial}\OperatorTok{$}\NormalTok{loadings, }\DataTypeTok{cutoff =} \FloatTok{0.5}\NormalTok{)}
\end{Highlighting}
\end{Shaded}

\begin{verbatim}
## 
## Loadings:
##                              MR1    MR2   
## vias_pavimentadas                         
## agua                          0.934       
## desague                       0.916       
## electrificacion               0.964       
## PBI_percapita                        0.888
## presupuestopublico_percapita         0.605
## 
##                  MR1   MR2
## SS loadings    2.869 1.369
## Proportion Var 0.478 0.228
## Cumulative Var 0.478 0.706
\end{verbatim}

\begin{itemize}
\tightlist
\item
  al ver que vias no aporta, SERIA MEJOR HACER DE NUEVO EL ANALISIS
  FACTORIAL SIN VIAS, o decides quedarte. Depende del diseño del trabajo
\item
  VAR. ACUMUAD:porcentaje de virbilidad, el factor uno 47\%, el factro 2
  es 70\%
\end{itemize}

Podemos visualizar los variables y su relación con las latentes creadas:

\begin{Shaded}
\begin{Highlighting}[]
\KeywordTok{fa.diagram}\NormalTok{(factorial)}
\end{Highlighting}
\end{Shaded}

\includegraphics{ejercicios_files/figure-latex/unnamed-chunk-13-1.pdf}

\hypertarget{paso-5-evaluamos-el-anuxe1lisis-factorial-exploratorio-solicitado-comprobamos-validez-af}{%
\subsection{Paso 5: Evaluamos el Análisis Factorial Exploratorio
solicitado (comprobamos validez
AF)}\label{paso-5-evaluamos-el-anuxe1lisis-factorial-exploratorio-solicitado-comprobamos-validez-af}}

\begin{itemize}
\tightlist
\item
  ¿La Raíz del error cuadrático medio corregida está cerca a cero? -
  Valores cercano a 0 indica un buen ajuste. Si no sale cerca al 0 eso
  significa que el análisis extraído no es suficientemente válido. En
  ese caso se REPORTA.
\end{itemize}

\begin{Shaded}
\begin{Highlighting}[]
\NormalTok{factorial}\OperatorTok{$}\NormalTok{crms}
\end{Highlighting}
\end{Shaded}

\begin{verbatim}
## [1] 0.08261519
\end{verbatim}

\begin{itemize}
\tightlist
\item
  ¿La Raíz del error cuadrático medio de aproximación es menor a 0.05?

  \begin{itemize}
  \tightlist
  \item
    0.032 es mayor a 0.05 entonces HAY PROBLEMA (fijarse en el primero)
  \end{itemize}
\end{itemize}

\begin{Shaded}
\begin{Highlighting}[]
\NormalTok{factorial}\OperatorTok{$}\NormalTok{RMSEA}
\end{Highlighting}
\end{Shaded}

\begin{verbatim}
##      RMSEA      lower      upper confidence 
##  0.3233485  0.1527133  0.5340611  0.9000000
\end{verbatim}

\begin{itemize}
\tightlist
\item
  ¿El índice de Tucker-Lewis es mayor a 0.9? Este índice tiende a 1 para
  modelos con muy buen ajuste, considerándose aceptables valores
  superiores a 0.90, aunque lo ideal sería valores mayores a 0.95.Si
  sale menor a 0.9 indica insuficiencia de validez. En ese caso se
  reporta.
\end{itemize}

\begin{Shaded}
\begin{Highlighting}[]
\NormalTok{factorial}\OperatorTok{$}\NormalTok{TLI}
\end{Highlighting}
\end{Shaded}

\begin{verbatim}
## [1] 0.5212291
\end{verbatim}

\begin{itemize}
\tightlist
\item
  ¿Qué variables aportaron mas a los factores?

  \begin{itemize}
  \tightlist
  \item
    aprota mas PBI y menos vias
  \end{itemize}
\end{itemize}

\begin{Shaded}
\begin{Highlighting}[]
\CommentTok{#sort es una funcion para ordenar de menor a mayor}
\KeywordTok{sort}\NormalTok{(factorial}\OperatorTok{$}\NormalTok{communality)}
\end{Highlighting}
\end{Shaded}

\begin{verbatim}
##            vias_pavimentadas presupuestopublico_percapita 
##                    0.1888712                    0.3676602 
##                      desague                         agua 
##                    0.8605013                    0.8758996 
##              electrificacion                PBI_percapita 
##                    0.9367019                    1.0082507
\end{verbatim}

\begin{itemize}
\tightlist
\item
  ¿Qué variables contribuyen a mas de un factor?
\end{itemize}

\begin{Shaded}
\begin{Highlighting}[]
\KeywordTok{sort}\NormalTok{(factorial}\OperatorTok{$}\NormalTok{complexity)}
\end{Highlighting}
\end{Shaded}

\begin{verbatim}
## presupuestopublico_percapita                         agua 
##                     1.006075                     1.007816 
##              electrificacion                      desague 
##                     1.017507                     1.053256 
##            vias_pavimentadas                PBI_percapita 
##                     1.098506                     1.516497
\end{verbatim}

\begin{Shaded}
\begin{Highlighting}[]
\CommentTok{#cada una de las variables, caada variable aporta a un factor.Pero tiene decimales, es deicr, aporta a 1.51 factores}
\CommentTok{#que variable aporta a mas de un factor: PBI}
\CommentTok{#que vairable aporta a menos de un factor: Presupuesto Publico}
\end{Highlighting}
\end{Shaded}

\hypertarget{paso-6-posibles-valores-proyectados}{%
\subsection{Paso 6: Posibles valores
proyectados}\label{paso-6-posibles-valores-proyectados}}

¿Qué nombres les darías?

Podemos crear un data set con sólo los factores creados

\begin{Shaded}
\begin{Highlighting}[]
\NormalTok{factorial_casos<-}\KeywordTok{as.data.frame}\NormalTok{(factorial}\OperatorTok{$}\NormalTok{scores)}
\KeywordTok{head}\NormalTok{(factorial_casos)}
\end{Highlighting}
\end{Shaded}

\begin{verbatim}
##          MR1        MR2
## 1 -1.2123710 -0.1879479
## 2  0.5244837  0.4357522
## 3 -0.5214367  0.7148069
## 4  1.2143593  0.7401245
## 5 -0.1678911 -0.2219972
## 6 -0.5750133 -0.2651315
\end{verbatim}

\begin{Shaded}
\begin{Highlighting}[]
\KeywordTok{summary}\NormalTok{(factorial_casos)}
\end{Highlighting}
\end{Shaded}

\begin{verbatim}
##       MR1                MR2         
##  Min.   :-1.48355   Min.   :-1.6099  
##  1st Qu.:-0.98449   1st Qu.:-0.4329  
##  Median :-0.04854   Median :-0.1260  
##  Mean   : 0.00000   Mean   : 0.0000  
##  3rd Qu.: 0.97628   3rd Qu.: 0.3127  
##  Max.   : 1.45470   Max.   : 4.0370
\end{verbatim}

o incluirlos en nuestro subset original

\begin{Shaded}
\begin{Highlighting}[]
\NormalTok{subdata}\OperatorTok{$}\NormalTok{factor1<-}\StringTok{ }\NormalTok{factorial_casos}\OperatorTok{$}\NormalTok{MR1}
\NormalTok{subdata}\OperatorTok{$}\NormalTok{factor2<-}\StringTok{ }\NormalTok{factorial_casos}\OperatorTok{$}\NormalTok{MR2}
\end{Highlighting}
\end{Shaded}

\begin{center}\rule{0.5\linewidth}{0.5pt}\end{center}

\hypertarget{ejercicio-2-realice-un-anuxe1lisis-factorial-con-todos-los-pasos-indicados-de-la-base-de-distritos-y-variables-del-sistema-poluxedtico-electoral}{%
\section{Ejercicio 2: Realice un análisis factorial con todos los pasos
indicados de la base de distritos y variables del sistema político
electoral}\label{ejercicio-2-realice-un-anuxe1lisis-factorial-con-todos-los-pasos-indicados-de-la-base-de-distritos-y-variables-del-sistema-poluxedtico-electoral}}

Abrimos la base de datos

\begin{Shaded}
\begin{Highlighting}[]
\KeywordTok{library}\NormalTok{(rio)}
\NormalTok{electoral<-}\StringTok{ }\KeywordTok{import}\NormalTok{(}\StringTok{"https://github.com/DataPolitica/salidas/raw/master/Data/ERM2010_Distrital.xlsx"}\NormalTok{)}
\KeywordTok{head}\NormalTok{(electoral)}
\end{Highlighting}
\end{Shaded}

\begin{verbatim}
##   UBIGEO   REGION   PROVINCIA    DISTRITO NEP 2010 NP 2010
## 1 010102 AMAZONAS CHACHAPOYAS    ASUNCION    2.292   1.465
## 2 010103 AMAZONAS CHACHAPOYAS      BALSAS    2.457   1.743
## 3 010104 AMAZONAS CHACHAPOYAS       CHETO    3.432   3.142
## 4 010105 AMAZONAS CHACHAPOYAS   CHILIQUIN    3.409   1.983
## 5 010106 AMAZONAS CHACHAPOYAS CHUQUIBAMBA    4.575   3.871
## 6 010107 AMAZONAS CHACHAPOYAS     GRANADA    2.942   2.147
##   HIPERFRACCIONAMIENTO 2010 VOLATILIDAD TOTAL 2010 CONCENTRACION 2010
## 1                     2.588                  0.002              0.841
## 2                     2.676                  1.000              0.850
## 3                     3.615                  1.000              0.659
## 4                     4.054                  0.033              0.649
## 5                     5.215                  0.071              0.566
## 6                     3.326                  0.027              0.787
##   HERFINDHAL HIRSCHMAN 2010 COMPETITIVIDAD 2010
## 1                     0.436               0.338
## 2                     0.407               0.216
## 3                     0.291               0.002
## 4                     0.293               0.265
## 5                     0.219               0.005
## 6                     0.340               0.124
\end{verbatim}

Realizamos un subset y nos quedamos con las variables que utilizaremos.
Lo llamaremos \textbf{subdata}:

\begin{Shaded}
\begin{Highlighting}[]
\NormalTok{subdata<-}\StringTok{ }\NormalTok{electoral[,}\KeywordTok{c}\NormalTok{(}\DecValTok{5}\OperatorTok{:}\DecValTok{11}\NormalTok{)]}
\end{Highlighting}
\end{Shaded}

\hypertarget{ejercicio-3-ide}{%
\section{Ejercicio 3: IDE}\label{ejercicio-3-ide}}

Abrimos la base de datos

\begin{Shaded}
\begin{Highlighting}[]
\NormalTok{ide<-}\StringTok{ }\KeywordTok{import}\NormalTok{(}\StringTok{"https://github.com/DataPolitica/salidas/raw/master/Data/IDE.sav"}\NormalTok{)}
\end{Highlighting}
\end{Shaded}

Realizamos un subset y nos quedamos con las variables que utilizaremos.
Lo llamaremos \textbf{subdata}:

\begin{Shaded}
\begin{Highlighting}[]
\NormalTok{subdata<-}\StringTok{ }\NormalTok{ide[,}\KeywordTok{c}\NormalTok{(}\DecValTok{6}\OperatorTok{:}\DecValTok{10}\NormalTok{)]}
\end{Highlighting}
\end{Shaded}

\hypertarget{ejercicio-4-democracy-index}{%
\section{Ejercicio 4: DEMOCRACY
INDEX}\label{ejercicio-4-democracy-index}}

Abrimos la base de datos

\begin{Shaded}
\begin{Highlighting}[]
\NormalTok{demo<-}\StringTok{ }\KeywordTok{import}\NormalTok{(}\StringTok{"https://github.com/DataPolitica/salidas/raw/master/Data/demo.xlsx"}\NormalTok{)}
\end{Highlighting}
\end{Shaded}

Realizamos un subset y nos quedamos con las variables que utilizaremos.
Lo llamaremos \textbf{subdata}:

\begin{Shaded}
\begin{Highlighting}[]
\KeywordTok{str}\NormalTok{(demo)}
\end{Highlighting}
\end{Shaded}

\begin{verbatim}
## 'data.frame':    167 obs. of  8 variables:
##  $ Pais                   : chr  "Noruega" "Islandia" "Suecia" "Nueva Zelanda" ...
##  $ PUNT                   : chr  "9.87" "9.58" "9.39" "9.27" ...
##  $ Pluralismo             : chr  "10.00" "10.00" "9.58" "10.00" ...
##  $ Funcionamiento_gobierno: chr  "9.64" "9.29" "9.64" "9.29" ...
##  $ Participacion_politica : chr  "10.00" "8.89" "8.33" "8.89" ...
##  $ Cultura_policita       : chr  "10.00" "10.00" "10.00" "8.13" ...
##  $ Derechos_civiles       : chr  "9.71" "9.71" "9.41" "10.00" ...
##  $ Cate                   : chr  "Democracia plena" "Democracia plena" "Democracia plena" "Democracia plena" ...
\end{verbatim}

\begin{Shaded}
\begin{Highlighting}[]
\NormalTok{demo[,}\OperatorTok{-}\KeywordTok{c}\NormalTok{(}\DecValTok{1}\NormalTok{,}\DecValTok{2}\NormalTok{,}\DecValTok{8}\NormalTok{)]=}\KeywordTok{lapply}\NormalTok{(demo[,}\OperatorTok{-}\KeywordTok{c}\NormalTok{(}\DecValTok{1}\NormalTok{,}\DecValTok{2}\NormalTok{,}\DecValTok{8}\NormalTok{)], as.numeric)}
\NormalTok{subdata<-}\StringTok{ }\NormalTok{demo[,}\KeywordTok{c}\NormalTok{(}\DecValTok{3}\OperatorTok{:}\DecValTok{7}\NormalTok{)]}
\NormalTok{subdata<-}\KeywordTok{na.omit}\NormalTok{(subdata)}
\KeywordTok{str}\NormalTok{(subdata)}
\end{Highlighting}
\end{Shaded}

\begin{verbatim}
## 'data.frame':    167 obs. of  5 variables:
##  $ Pluralismo             : num  10 10 9.58 10 10 9.58 10 9.58 10 9.58 ...
##  $ Funcionamiento_gobierno: num  9.64 9.29 9.64 9.29 8.93 7.86 9.29 9.64 8.93 9.29 ...
##  $ Participacion_politica : num  10 8.89 8.33 8.89 8.33 8.33 8.33 7.78 7.78 7.78 ...
##  $ Cultura_policita       : num  10 10 10 8.13 8.75 10 9.38 8.75 8.75 9.38 ...
##  $ Derechos_civiles       : num  9.71 9.71 9.41 10 9.71 10 9.12 10 10 9.12 ...
\end{verbatim}

\end{document}
